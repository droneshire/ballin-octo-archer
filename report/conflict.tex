\subsection{Conflict Detection}

To be effective at preventing traffic accidents, the controller for an
automated intersection must have an accurate method for detecting
collisions between individual vehicles before they arrive within minimum
stopping distance of the intersection.  Since the control problem
imposes hard real-time requirements, the primary concern was performance
that scaled gracefully with the number of participating vehicles.

As always, a balance must be found between fidelity and tractability.
For the highest accuracy, trajectories can be represented as
three-dimensional curves, with coordinates consisting of a temporal
component as well as spatial position.  Preventing collisions is
therefore equivalent to enforcing a lower bound on the Euclidean
distance between two such curves.  However, this method demands an
unrealistic amount of information about vehicular physics than is
obtainable during runtime.  We desire an alternative method that is
less numerically intensive, so it can be repeatedly applied as traffic
conditions evolve, but whose approximations do not compromise safety.

Focusing on the locations at which collisions are most likely to occur,
we begin with the concept of a \emph{traffic conflict point}, defined
as the point at at which multiple traffic flows may cross or merge.
As per normal regulations, we require that all vehicles adhere to
designated lanes when traversing the intersection, so that trajectories
can be bounded exactly.  Thus, the set of conflict points are determined
by the intersections between curves representing all legal paths of
vehicles.

We assume the following in our reference intersection model:
\begin{enumerate}
\setlength{\itemsep}{0pt}
\setlength{\parskip}{0pt}
\item Four-way symmetric intersection with orthogonal arms and
	perfect alignment
\item Open space without obstructions (e.g., pedestrians)
\item The total number of lanes at any exit approach equals the
	total number of lanes at any entrance approach.
\item Traffic lanes at an exit approach can only serve as the
	exit of lanes with the same lane number relative to
	the center line.
\end{enumerate}

Straight paths are simply modeled as vertical and horizontal lines.
The left-turn paths are circular arcs that circumscribe three control
points at the entrance, midpoint, and exit of the turn; the midpoints
are chosen to avoid conflicts with oncoming left-turn traffic.
For simplicity, U-turns are disallowed.  The right-turn paths are
circular arcs which subtend a right angle.\footnote{The actual path
equations are viewable on the presentation slides submitted along with
this report.}

In reality, vehicles are not point masses -- they occupy area.
We desire to enforce a minimum separation between vehicles traveling
on conflicting paths.  A \emph{traffic conflict zone} is defined as the
segment of a given path within a certain distance of a conflict point.
The initial and terminal points of a conflict zone are expressed as an
ordered pair of offsets/displacements along the curve, relative to the
start of the path.

The collision detection algorithm is therefore:
\begin{enumerate}
\setlength{\itemsep}{0pt}
\setlength{\parskip}{0pt}
\item Determine the set of conflict zones along the path.
\item Numerically integrate to determine when endpoints of conflict
	zones are crossed, yielding a set of time intervals.
\item For each conflict zone, check whether the corresponding time
	interval overlaps with that of any other active vehicle.
\end{enumerate}

It is important to note that conflict zones are static and therefore
need only to be calculated once during initialization.  Runtime does
not require knowledge about curvature of paths; only linear
displacements matter.  Reservations for a vehicle are thus scheduled
as time intervals during which it has exclusive access to the conflict
zones along its path.
