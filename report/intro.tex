Once limited to the realm of science fiction, autonomous consumer vehicles, or driverless cars, present a very exciting opportunity for the application of interdisciplinary research from artificial intelligence, robotics, control theory, and embedded systems. Since the late 2000s, numerous advances have been made in the field. Prototypes have been researched and/or tested by Toyota, Nissan, and Audi. Three U.S. states have passed laws permitting autonomous vehicles. A world full of driverless cars presents a host of benefits, but much work is yet to be done before this becomes a reality.\\
The full assimilation of autonomous vehicles in daily life involves solving many subproblems such as ensuring safety, implementing error-free and robust control mechanisms, and building corresponding infrastructure. This includes the design of intersections that support and optimize for autonomous vehicles.  Traditional, four way intersections will need to be revamped: red-yellow-green switching will be rendered mostly obsolete, and traffic will be continuously directed at a much finer scale by an autonomous controller agent.

In this paper, we propose a model for such an intelligent, automated intersection of two lanes on each side, to facilitate efficient routing of a sampling of vehicles that are mostly autonomous. Our model, unlike some existing research, advocates for a centralized controller instead of a P2P protocol. This centralized controller will ensure safety (i.e. no collisions) and fairness. We will formalize the communications between the car and the central controller, provide the mathematical groundwork for the geometry of our intersection and the paths of the vehicles and analyze our results. We also provide a demonstrative model of our system using Ptolemy II. With our basic model we hope to provide groundwork for the development of a more robust solution. 
