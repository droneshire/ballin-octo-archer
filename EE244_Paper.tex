\documentclass[10pt]{article}

\usepackage{amsmath}
\usepackage{relsize}
\usepackage{amssymb}
\usepackage{amsthm}
\usepackage{stmaryrd}
\usepackage{cancel}
\usepackage{wasysym}
\usepackage[usenames,dvipsnames]{color}
\usepackage{xspace}
\usepackage{url}
\usepackage{verbatim}
\usepackage{proof}
\usepackage{fullpage}
\usepackage{xr-hyper}
\usepackage[colorlinks,urlcolor=MidnightBlue,linkcolor=MidnightBlue,citecolor=MidnightBlue,backref]{hyperref}
\newcommand{\tab}{\hspace*{2em}}
\usepackage{graphicx}
\usepackage{amsmath}
\usepackage{newlfont}
\usepackage{listings}
\usepackage{pgf}
\usepackage{tikz}
\usepackage[latin1]{inputenc}
\usepackage{wrapfig}
\usetikzlibrary{arrows,automata,shapes}
\usepackage{float}
\usepackage{caption}
\usepackage{subcaption}
\usepackage{amssymb}
%\floatstyle{boxed}
\restylefloat{figure}
\setlength{\parindent}{0cm}


\title{Automated Intersection Modelling}
\author{Albert Ou, Praagya Singh, & Ross Yeager}
\date{May 13, 2013}


\begin{document}

\maketitle
\section{Introduction and Problem Definition}

\section{Previous Work}

\section{Algorithms and Models Devised}

\subsection{Vehicles}
In our autonomous intersection model, there are two types of vehicles that are defined: autonomous vehicles and human driven vehicles.  
Both vehicle types take on the same physical characteristics in our model; however, they differ in behavioral capabilities.  
Typically, cars are classified into many different phycial categories.  
For simplicity and tractibility, our model utilizes the most common type of vehicle, the passenger car (denoted as type P vehicle).  
Minimum characteristics of type P cars require a design radius of 24 ft with an minimum inside radius of 14.4 ft. from the inner edge of the lane \cit{Bureau}.


\subsection{Intersection}
The intersection analyzed in this study was a typical intersection consisting of four perpendicular 4-lane roadways.   As specified in standard roadway construction specifications, each of these roadways have an intersection spacing distance of 375 ft \cit{Bureau}.
This was in order to provide a sufficient distance in which vehicles can properly maneuver in standard vehicle behaviors. 

\section{Technical Challenges}

\section{Results}

\section{Team Roles}

\subsection{Albert Ou}

\subsection{Praagya Singh}

\subsection{Ross Yeager}

\section{Conclusion}

\subsection{Course Application}

\subsection{Feedback}

\begin{thebibliography}{100}
\bibitem[1]{Bureau} "Bureau of Local Roads and Streets Manual." Illinois Department of Transportation. Illinois Department of Transportation, last modified October 2008, accessed May, 2013, 2013, http://www.dot.il.gov/blr/manuals/blrmanual.html.
\bibitem[2]{Dresner} Dresner, K. and P. Stone. 2006. "Traffic Intersections of the Future." AI Magazine, 1593-1593-1596.
\bibitem[3]{Lu} Lu, J., S. Chen, X. Ge, and F. Pan. 2012. "A Programmable Calculation Procedure for Number of Traffic Conflict Points at Highway Intersections." Journal of Advanced Transportation 46 (10).
\bibitem[4]{Osiecki} Osiecki, L., Canning, K. and Scarborough, W. "Delaware Department of Transportation Road Design Manual." Delaware.gov. Delaware Department of Transportation, last modified October 1, 2004, accessed May/2013, 2013, http://www.deldot.gov/information/pubs_forms/manuals/road_design/.
\bibitem[5]{Wolshon} Wolshon, B. 2004. Toolbox on Intersection Safety and Design. The Institute of Transportation Engineers:.
\end{thebibliography}

\end{document}
