\documentclass{beamer}
\usepackage{amsmath}
\usepackage{graphicx}

\usetheme{Copenhagen}
\usecolortheme{beaver}
\usefonttheme[onlymath]{serif}
\setbeamertemplate{itemize items}[default]

\begin{document}

\section{Introduction}

\begin{frame}{Automotive Intersections}
INSERT PICTURE OF CRAZY INTERSECTION HERE
\end{frame}

\begin{frame}{Autonomous Intersections}
\begin{itemize}
\item Future roadways face congestion issues due to increasing volume
\item Autonomous cars will become much more prevalent in future
\item GOAL: Model an intelligent, autonomous vehicle-based intersection system with a centralized controller intead of P2P protocol
\end{itemize}
\end{frame}

\section{Objectives and Assumptions}

\subsection{Original Objectives}

\begin{frame}{Original Vehicle Model}
Originally, we assumed an intersection environment that contained both
	human and autonomously controlled vehicles where:\\
\begin{itemize}
\item Human controlled vehicles respond only to $START$ and 
	$STOP$ commands by the controller
\item Autonomous vehicles repond to $START$, $STOP$, $CHNG$, and $SLWDN$
\item Controller analyzes all vehicles in system and sends out commands to 
	optimize throughput
\item The intersection is a hybrid system containing both discrete and continuous components
\end{itemize}
\end{frame}

\begin{frame}{Controller Knowledge}
Controller has continuous knowledge of certain vehicle properties at all times within the intersection environment due to 
	\emph{embedded wireless communication devices} in each vehicle continously sending:\\
\begin{itemize}
\item Current Speed
\item Route Planned
\item Entrance Lane
\item Location
\end{itemize}
\end{frame}

\subsection{Objective Changes}

\begin{frame}{Deviations from Original Plan}
Changes in Intersection System Model:
\begin{itemize}
\item Shift of focus from optimization towards correct modelling
\item $a(t) = 0$
\item Changes in velocity can occur instantaneously
\item System contains autonomous vehicles only
\item Vehicle reaction to controller command is instantaneous
\item Intersection turns have no effect on velocity
\end{itemize}
\end{frame}

\begin{frame}{Reasons for Changes}
\begin{itemize}
\item Optimization of the system may be intractable for application
\item Human controlled vehicles would be a rarity in the described system
\item Constant velocity model reasonable within limited intersection distances
\item Limitations with dynamic actor instantiation in Ptolemy modelling
\item Learning curve involved with modelling tool (Ptolemy)
\end{itemize}
\end{frame}

\begin{frame}{Modelling Objectives}
\begin{block}{Model Objectives}
Let $c$ denote the event in which a collision event has occured at time $t$, and let $e$ denote the event in which a car, $v$, has entered a route within the intersection.\\
\begin{itemize}
\item Safety:  $\forall t: \neg Gc$
\item Fairness:  $ \forall v: Fe$
\end{itemize}
\end{block}
\end{frame}


\section{Vehicle Model}

\subsection{Tools and Vehicle Assumptions}

\begin{frame}{Modelling Tool}
Modelling Tool: Ptolemy II Version 8.0.1\\\\

INSERT PIC HERE

%%%%\centering\includegraphics[width=0.8\linewidth]{INSERT ptolemy.eps HERE}
\end{frame}

\begin{frame}{Assumptions}
\begin{block}{Vehicle Properties}
Properties exhibited by each vehicle in the model:
\begin{itemize}
\item Instantaneous change in velocity
\item Constant velocity within intersection
\item Velocity (mph): $V \rightarrow V \in [20,40]$
\item Route intention constant throughout intersection traversal
\item Randomized vehicle entry times
\item Receives velocity change or delayed start time from controller
\end{itemize}
\end{block}
\end{frame}

\subsection{Ptolemy Vehicle Model}

\begin{frame}{Intersection Model}
INSERT PIC HERE
%%%%\centering\includegraphics[width=0.8\linewidth]{INSERT intersectionSystem.eps HERE}
\end{frame}

\begin{frame}{Modelling Continuous Vehicle Flow}
Ptolemy limitations prevent dynamic instantiation of actors.\\
\begin{block}{Multi-Instance Composite Actor}
\begin{minipage}{0.45\linewidth}
\begin{itemize}
\item Vehicle instances accessed by ID encoding
\item Create $N$ vehicle Modal Model instances
\item Generate random:
\begin{itemize}
\item Intersection entry time: $1-45$ secs
\item Initial velocity (m/s): $V_0 \in [20,40]$
\item Direction: $N,S,E,W$
\item Route: $L,S1,S2,R$
\end{itemize}
\end{itemize}
\end{minipage}
\hfill
\begin{minipage}{0.45\linewidth}

\begin{itemize}
\item Implement as many instances of the car model as desired
\item Interacts with Python-based controller
\item Car can "re-enter" system only after it exits (re-use vehicles 
	vs. generating new vehicles)
\end{itemize}
\end{minipage}
\end{block}
\end{frame}


\begin{frame}{Vehicle Modelling}
INSERT PIC HERE
%%%%\centering\includegraphics[width=0.8\linewidth]{INSERT multiinstanceactor.eps HERE}
\end{frame}

\begin{frame}{Vehicle FSM Actor}
\begin{block}{States}
\begin{itemize}
\item $RUN$
\item $IDLE$
\item $ENTER$
\item $BOOK$
\item $GO$
\item $WAIT$
\end{itemize}
\end{block}
\end{frame}

\begin{frame}{Vehicle FSM Actor}
INSERT PIC HERE
%%%%\centering\includegraphics[width=0.8\linewidth]{INSERT fsmactor.eps HERE}
\end{frame}

\begin{frame}{ENTER State}
\begin{minipage}{0.45\linewidth}
Abstraction:
\begin{itemize}
\item Equivalent to pre-conflict zone straightaway 
\item Initial velocity used to calculate time until intersection
\item Stops at intersection if instructions not received from controller
\end{itemize}
\end{minipage}
\hfill
\begin{minipage}{0.45\linewidth}
Guard(s);
\begin{itemize}
\item $t_{GLOBAL} > t_{ENTER}$
\end{itemize}
\end{minipage}
\end{frame}

\begin{frame}{ENTER State}
INSERT PIC HERE
%%%%\centering\includegraphics[width=0.8\linewidth]{INSERT enterState.eps HERE}
\end{frame}

\begin{frame}{GO State}
\begin{minipage}{0.45\linewidth}
Abstraction:
\begin{itemize}
\item Equivalent to the approach through the intersection
\item Tracks when the vehicle leaves the system
\item Stops at intersection if instructions not received from controller
\end{itemize}
\end{minipage}
\hfill
\begin{minipage}{0.45\linewidth}
Guard(s);
\begin{itemize}
\item $t_{GLOBAL} > t_{STRAIGHTAWAY}$
\end{itemize}
\end{minipage}
\end{frame}


\begin{frame}{GO State}
INSERT PIC HERE
%%%%\centering\includegraphics[width=0.8\linewidth]{INSERT gostate.eps HERE}
\end{frame}

\section{Controller Model}

\subsection{Controller Properties}

\begin{frame}{Controller Properties}
\begin{block}{Controller}
\begin{itemize}
\item Lack of multiport output (serial communication)
\item Delayed feedback to vehicle
\begin{itemize}
\item Prevent a zeno system
\item Models communication delay
\end{itemize}
\item Receives requests and sends commands to system vehicles
\item Inputs to controller: $velocity$, $booking$, $entertime$ 
\item Python script actor
\end{itemize}
\end{block}
\end{frame}

\subsection{Ptolemy Model}

\begin{frame}{Ptolemy Model}
INSERT PIC HERE
%%%%\centering\includegraphics[width=0.8\linewidth]{INSERT intersectionSystem.eps HERE}
\end{frame}

\subsection{Controller Algorithm}
\begin{frame}{Algorithm Overview}
\begin{lstlisting}
PLACE CODE HERE
\end{lstlisting}
\end{frame}


\begin{frame}{Algorithm Walkthrough}
Show Ptolemy Model
\end{frame}

\section{Technical Challenges}

\begin{frame}
\begin{block}{Top 4 Challenges}
\begin{itemize}
\item Continuous car flow given static number of vehicles
\item Delivering controller commands to specified car instance
\item Continuous time solver time step evaluation issues
\item Hybrid system causality issues
\end{itemize}
\end{block}
\end{frame}


\section{Summary}

\begin{frame}{Summary}

\begin{block}{Project Conclusions}
\begin{minipage}{0.45\linewidth}
Achieved:
\begin{itemize}
\item Functional hybrid model 
\item Continuous vehicle flow with static vehicle set
\item Centralized responsive controller
\item Rigorous intersection path definitions
\item Multiplexed communication to vehicles
\end{itemize}
\end{minipage}
\hfill
\begin{minipage}{0.45\linewidth}
Further Work:
\begin{itemize}
\item Further optimize controller response algorithm
\item Introduce vehicle acceleration into the model
\item Add support for human cars
\end{itemize}
\end{minipage}
\end{block}
\end{frame}

\end{document}
